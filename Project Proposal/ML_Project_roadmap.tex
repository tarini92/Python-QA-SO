\documentclass{article}
\usepackage[utf8]{inputenc}
\usepackage[document]{ragged2e}
\usepackage{algpseudocode}
\usepackage[]{algorithmicx}
\usepackage{amsmath}
\usepackage{amsthm}
\usepackage{amssymb}
\usepackage[]{listings}
\usepackage{graphicx}
\usepackage{hyperref}
\usepackage{flafter}
\usepackage{subfig}
\usepackage{dsfont}
\graphicspath{ {images/} }

\begin{document}

\begin{titlepage}
	\centering
	\includegraphics[width=0.15\textwidth]{IIIT-B_logo.jpg}\par\vspace{1cm}
	{\scshape\LARGE International Institute of Information Technology, Bangalore \par}
	\vspace{1cm}
	{\scshape\Large Project Proposal\par}
	{\Large  CS/DS 706 Machine Learning\par}
	\vspace{1.5cm}
	{\huge\bfseries Python Stackoverflow QA Analysis\par}
	\vspace{2cm}
	{\Large\itshape Akanksha Dwivedi - MT2016006\par}
	{\Large\itshape Tarini Chandrashekhar - MT2016144\par}
	\vfill
	Instructor : \par
	Prof. Dinesh Babu J 

	\vfill

% Bottom of the page
	{\large \today\par}
\end{titlepage}

\newpage

\tableofcontents

\newpage
\justify

\section{Brief Description}

This project aims at utilising natural language processing and exploratory analytics on a dataset consisting of Python question and answers on Stack Overflow, to design a novel course structure for teaching Python. 

\subsection{Problem Formulation}

\section{Dataset}
The dataset consists of full text of questions and answers from StackOverflow, that are tagged with the python tag, useful for natural language processing and community analysis.  
\par
This is organized as three tables i.e three .csv files:
\begin{itemize}
\item \textbf{Questions} contains the title, body, creation date, score, and owner ID for each Python question.

\item \textbf{Answers}  contains the body, creation date, score, and owner ID for each of the answers to these questions. The ParentId column links back to the Questions table.

\item \textbf{Tags} contains the tags on each question besides the Python tag.
\end{itemize}




\section{Proposed Plan of execution}



\section{Main Challenges}


\section{Learning Objectives}



\end{document}
